% Resume Template by Cristian Palma Foster
\documentclass[10pt]{article}
\usepackage[utf8]{inputenc}
\usepackage[left=2cm,top=1.5cm,right=2cm,bottom=2.5cm]{geometry} 
% Manage images
\usepackage{graphicx}
% Place images
\usepackage{wrapfig}
% Make columns
\usepackage{multicol}
% Resize titles
\usepackage{sectsty}
\usepackage{hyperref}
\usepackage{xcolor}
% Erase page numbers
\pagenumbering{gobble}
% Well spaced paragraphs
\setlength{\parskip}{-5pt}
\setlength\parindent{0pt}
% Set the section size
\sectionfont{\fontsize{12}{15}\selectfont}
% Well spaced lines in tables
\renewcommand{\arraystretch}{1.3}

% Contact Information
\newcommand{\contactinformation}[1]{
    \begin{center}
        \vspace{1cm}
        \huge\textbf{Cristian Palma - #1}\\
        \vspace{0.3cm}
        \normalsize
        \href{mailto:cristian.palma.foster@gmail.com}{cristian.palma.foster@gmail.com},
		\href{http://www.linkedin.com/in/cristianpalmaf}{linkedin.com/in/cristianpalmaf},
		\href{http://www.github.com/elcrislwh}{github.com/elcrislwh}
        \\
    \end{center}
}

\newenvironment{concept}[1]{
    \vspace{0.1cm}
    \begin{large} \textcolor{gray}{\textbf{#1:}}\end{large}\\
    \vspace{-0.1cm}
    
}{}

\newcommand{\job}[2]{
    \large \textbf{#1} \hfill \small \textcolor{gray}{\textit{#2}}\\
}
\newcommand{\data}[2]{
    \small #1 \hfill \small \textcolor{gray}{\textit{#2}}\\
}
\newcommand{\desc}[1]{
    \small #1 
    \vspace{0.2cm}\\
}

\newenvironment{map}[1]{
    \concept{#1}
    
    \hspace{-0.3cm}
    \begin{tabular}{lr}
}{ 
    \end{tabular}\\
}

\newcommand{\entry}[2]{
    \small #1 & \small #2\\
}
\begin{document}

\contactinformation{Software Engineer}

\begin{concept}{Educación}
\data{MS en Matemáticas Aplicadas en Algoritmos Combinatoriales, Universidad de Chile.}{Chile, 2022 - 2023}
\data{MS+BS en Ingeniería Matemática y Ciencias de la Computación, École CentralSupélec.}{Francia, 2019 - 2021}
\data{BS en Ingeniería en Matemática y Ciencias de la Computación, Universidad de Chile.}{Chile, 2017 - 2019}
\end{concept}

\begin{concept}{Experiencia Laboral}
\job{Research Assistant en Centro de Modelamiento Matemático}{Chile, 2022}
\desc{Desarrollados esquemas de aproximación a tiempo polinomial para problemas de Machine Scheduling.}
\job{Software Engineer Intern en Bigblue}{Francia, 2021}
\desc{Resolución de Problemas y desarrollo \textbf{Backend} en \textbf{Golang} con \textbf{FoundationDB} y \textbf{Google Cloud} en logística e-commerce.\\ 
Habilitadas entregas verdes \textbf{integrando} un Transportador Ecológico al \textbf{Transportation Management System}.\\
% Permitido lanzamiento en España integrando un Transportador Local.\\
Hecho el \textbf{Sistema de Facturación} escalable para facturar múltiples países extendiendolo con configuraciones por país.\\
Aumentada la \textbf{Robustez} del \textbf{Sistema de Facturación} implementando un Validador Semántico y un \textbf{Verificador de Duplicados} para facturas manuales.\\
Disminuido el tiempo de facturación en 50\% automatizando el \textbf{Batch Import} de líneas de facturación.}
\job{Práctica de Verano en Amazon}{Francia, 2020}
\desc{Ayudado al Warehouse Manager y Recoger y Empacar pedidos.}
\job{Desarrollador de la Asoc Internacional de CentraleSupélec}{Francia, 2020}%del Tandem de Lenguas
\desc{Emparejado 47\% más gente con respecto al año pasado para reuniones de conversación multilingües implementando un \textbf{Algoritmo de Matching} en \textbf{Python}.}%Reducido Tiempo de Calculo a milisegundos
% \job{Research Intern en Sondra Signal Processing Lab}{Francia, 2020}
% \desc{Aumentado Retorno sobre Inversión en 50\% usando un Portafolio de Mínima Varianza aplicando un método de Selección de Orden para \textbf{Procesamiento de Imágenes y Finanzas} en \textbf{Matlab}.}
\end{concept}

\begin{concept}{Enseñanza Auxiliar}
\data{\textbf{Algoritmos Combinatoriales} y Programación Lineal Mixta en Universidad de Chile}{2022}
\data{\textbf{Algoritmos y Estructuras de Datos} en Universidad de Chile}{2018}
\end{concept}

\begin{concept}{Proyectos}
\data{\textbf{Frontend} dev de Sitio Planificador de Maratones de Series en \textbf{JavaScript} usando \textbf{API} de TMDB.}{2021}%Binger
\data{\textbf{Frontend} dev de Sitio de Pronostico del Tiempo en \textbf{JavaScript} usando \textbf{API} de OpenWeather y \textbf{Node.js}.}{2021}%Aurora
% \data{Reevaluación de Seguros usando \textbf{Clustering} y \textbf{Optimización Paramétrica} en \textbf{Matlab}.}{2021}%con Generali y CentraleSupélec
% \data{Control de Ruido usando \textbf{Optimización Paramétrica} de EDP en \textbf{Python}.}{2020}%con Onera y CentraleSupélec
\data{\textbf{Backend} dev de Sitio de Recomendación de Películas con Amazon \textbf{DynamoDB}.}{2020}%with Theodo and CentraleSupélec
\data{\textbf{Algoritmo de Maximización de Propagación} en \textbf{Python} para Campaña de Marketing Viral.}{2020}%con Artefact y CentraleSupélec
\data{\textbf{Full Stack} dev de Prototipo de \textbf{Red Social} basada en la Geoproximidad en Facebook \textbf{Hackathon} Chile.}{2019}%on JavaScript
\data{Desarrollo y Armado de Prototipo de Robot Guía en \textbf{Java} usando Lego Mindstorms.}{2018}%Ruteador
% \data{Optimización de Multiplicación de Matrices.}{2018}
% \data{Codificación de Huffman.}{2018}
\data{Desarrollo y Armado de Prototipo de Robot Detector de Partes Defectuosas en \textbf{Clang} usando Arduino.}{2017}
\data{Replicación Básica de Juego de Pokémon en \textbf{Python}.}{2017}
% \data{Calculador de Expersiones y Derivadas.}{2017}
\end{concept}

\begin{concept}{Honores}
% \data{Reconociemiento como Alumno Destacado de Ingenriería Matemática, Universidad de Chile.}{2017 - 2018}
\data{Beca de Excelencia Eiffel otorgada por el Ministerio de Europa para Asuntos Extranjeros.}{2019 - 2021}
\data{Primer Lugar en Ranking de la Promoción en Ingeniería, Universidad de Chile.}{2018}
% \data{Reconociemiento como Alumno Destacado de Plan Común de Ingenriería, Universidad de Chile.}{2017 - 2018}
\data{Beca de Excelencia Andrés Bello otorgada por la Universidad de Chile.}{2017 - 2022}
% \data{Octavo lugar en  Ranking de Admisión en Ingeniería, Universidad de Chile.}{2017}
% \data{Doble Puntaje Nacional, Matemáticas y Ciencias mención Física en PSU.}{2016}
% \data{Reconocimiento al Mejor Rendimiento Académico en el Colegio Seminario Padre Alberto Hurtado.}{2016}
% \data{Medalla de Oro en la Olimpiada Nacional de Matemáticas de la SOMACHI, Chile.}{2016}
% \data{Medalla de Plata en la Olimpiada Nacional de Matemáticas de la SOMACHI, Chile.}{2015}
% \data{Medalla de Oro en la Olimpiada Regional de Física de la SOCHIFI Bío-Bío, Chile.}{2015}
% \data{Medalla de Bronce en la Olimpiada Nacional de Matemáticas de la SOMACHI, Chile.}{2014}
% \data{Medalla de Plata en la Olimpiada Nacional de Química de la SCHQ, Chile.}{2014}
% \data{Medalla de Bronce en la Olimpiada Nacional de Matemáticas de la SOMACHI, Chile.}{2013}
% \data{Participación en la Olimpiada Nacional de Matemáticas de la SOMACHI, Chile.}{2012}
\end{concept}

\vspace{-0.3cm}
\setlength{\columnsep}{4cm}
\begin{multicols}{2}
\begin{map}{Informática}
% \entry{Escritura}{MS Office, Jupyter y \LaTeX.}
\entry{Programación}{Python, Java, Clang, Golang y GitHub.}
\entry{Web}{HTML, CSS, JavaScript, Node.js y React.}
\entry{Bases de Datos}{DynamoDB, FoundationDB y MySQL.}
\entry{Cloud}{Google Cloud y Amazon Web Services.}
\end{map}

\begin{map}{Idiomas}
\entry{Inglés}{Avanzado.}
\entry{Francés}{Avanzado.}
\entry{Alemán}{Elemental.}
\entry{Español}{Nativo.}
\end{map}
\end{multicols}
\vspace{-0.4cm}

\begin{concept}{Free Time}
Volleyball, Bodybuilding, Bailar, Correr, Kung Fu and \textbf{Programación Competitiva}. % Kung Fu (faja morada, aprendido palo y sable), Correr (terminado Paris 10km Run en 52min).
\end{concept}

\end{document}